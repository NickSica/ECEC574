\documentclass{article}
\usepackage[left=0.5in, right=0.5in, top=0.75in, bottom=0.75in]{geometry}
\usepackage{multirow}

\title{IC Compiler Assignment}
\author{Nicholas Sica}
\date{March 22, 2023}

\begin{document}
\maketitle

There was not much done in terms of optimizing or constraining the area and
power. The only additional constraint for area was setting the max area to zero
to get a minimum area design. One oddity I noted while running the scripts is
the fact that the verify\_zrt\_route command reported violations while the
quality of results report reported no violations. After discussion, I decided
the quality of results report was more trustworthy. The overall trend for the
individual tables is for the numbers to increase in magnitude as the designs get
more complex. When it comes to comparing the pre-layout and post-layout results,
that is where it gets a bit interesting. In some cases, the cell count changed
which shows how the different steps in compiling the circuit ended up adding or
removing cells. This allowed for the worst negative slack and total negative
slack to end up in a good place where previously they would be failing. It is
also interesting how the biggest design ended up with a bigger area despite
having overall less cells. One possible explanation for that is the routing
taking up more area than expected. Power numbers seem relatively the same as
they were before except in the two largest designs. The clock skew is also below
100ps on all of the designs with the highest being 0.0347ns or 34.7ps. The clock
buffers are non-existent for the most of the designs except the two biggest
while each design has 3-4x more clock sinks than the previous design above it.

Overall, this was a success with no design rule check violations, no slack
errors, and designs that looked good when inspected in the GUI. If I had more
time to do some testing, I would try it with the faster clock frequency that we
were having issues with and try to make the best version of the design I
could. To run the scripts, the bash files provided in the respective directories
will run each design. If you run it manually it will only run the design
specified in the set\_env.tcl file. The output directories are under their
respective folders(reports for all the different reports, output for the
extracted designs, etc) with a folder for each design.


\begin{table}[h]
\centering
\caption{Pre-layout and post-layout simulation results.}
\begin{tabular}{ |c|c|c|c|c|c|c|c|c|c|c|c|c| }
    \hline
    \multirow{2}{*}{Design} & \multicolumn{5}{|c|}{Pre-Layout Results} & \multicolumn{6}{|c|}{Post-Layout Results} \\
    \cline{2-12}
    & Area & Power & WNS & TNS & Cell Count & Area & Power & WNS & TNS & Cell Count & DRC \\
    \hline
    s386.v & 185.22 & 5.6739 & 0.42 & 0.00 & 64 & 185.22 & 5.8457 & 0.24 & 0.00 & 64 & 0 \\
    \hline
    s1238.v & 1007.51 & 29.4925 & -0.03 & -0.06 & 368 & 959.66 & 29.1348 & 0.17 & 0.00 & 342 & 0 \\
    \hline
    s9234.v & 1994.45 & 45.0913 & 0.09 & 0.00 & 495 & 1993.57 & 45.9801 & 0.10 & 0.00 & 493 & 0 \\
    \hline
    s15850.v & 6878.56 & 157.3402 & -0.06 & -1.26 & 1652 & 6870.20 & 220.3384 & 0.12 & 0.00 & 1666 & 0 \\
    \hline
    s35932.v & 24807.70 & 537.0980 & -0.04 & -0.39 & 5002 & 25783.95 & 794.4180 & 0.17 & 0.00 & 4887 & 0 \\
    \hline
\end{tabular}
\end{table}

\begin{table}[h]
\centering
\caption{Clock information.}
\begin{tabular}{ |c|c|c|c| }
    \hline
    Design & \# Clock Sinks & \# Clock Buffers & Global Clock Skew \\
    \hline
    s386.v & 6 & 0 & 0.0001 \\
    \hline
    s1238.v & 18 & 0 & 0.0007 \\
    \hline
    s9234.v & 125 & 0 & 0.0072 \\
    \hline
    s15850.v & 442 & 75 & 0.0439 \\
    \hline
    s35932.v & 1728 & 81 & 0.0347 \\
    \hline
\end{tabular}
\end{table}

\end{document}
