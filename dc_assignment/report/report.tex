\documentclass[journal, onecolumn]{IEEEtran}

\title{Design Compiler Synthesis Assignment}
\author{Nicholas Sica}
\date{March 8, 2023}

\begin{document}
\maketitle
\begin{table}
  \centering
  \caption{100MHz DC logic synthesis results.}
  \begin{tabular}{ |c|c|c|c|c|c|c| }
    \hline
    Design & Area & Power & WNS & TNS & Cell Count & Num of violating paths \\
    \hline
    s386.v & 185.22 & 5.6739 & 0.42 & 0.00 & 64 & 0 \\
    \hline
    s1238.v & 1007.51 & 29.4925 & -0.03 & -0.06 & 368 & 3 \\
    \hline
    s9234.v & 1994.45 & 45.0913 & 0.09 & 0.00 & 495 & 0 \\
    \hline
    s15850.v & 6878.56 & 157.3402 & -0.06 & -1.26 & 1652 & 20 \\
    \hline
    s35932.v & 24807.70 & 537.0980 & -0.04 & -0.39 & 5002 & 41 \\
    \hline
  \end{tabular}
\end{table}
\begin{table}
  \centering
  \caption{500MHz DC logic synthesis results.}
  \begin{tabular}{ |c|c|c|c|c|c|c| }
    \hline
    Design & Area & Power & WNS & TNS & Cell Count & Num of violating paths \\
    \hline
    s386.v & 302.85 & 11.8507 & -2.59 & -20.63 & 121 & 13 \\
    \hline
    s1238.v & 1579.44 & 62.3031 & -3.37 & -56.73 & 655 & 32 \\
    \hline
    s9234.v & 4201.83 & 117.5054 & -2.53 & -256.46 & 1230 & 187 \\
    \hline
    s15850.v & 9627.03 & 257.2878 & -4.05 & -1138.71 & 2559 & 607 \\
    \hline
    s35932.v & 27369.22 & 790.5770 & -2.81 & -3166.87 & 5730 & 2048 \\
    \hline
  \end{tabular}
\end{table}
There was not much done in terms of optimizing or constraining the area and power. The only additional constraint for area was setting the max area to zero to get
a minimum area design. The overall trend for the individual tables is for the numbers to increase in magnitude as the designs get more complex.
This does not hold as true for the 100MHz designs as the worst negative slack and total negative slack values depend on if it can correctly
optimize the few nets that are violating. While there are violations for some of the 100MHz designs, this would be fine to continue forward
with the physical design to see if the worst negative slack and total negative slack can be optimized for in those stages. The 500MHz clock
failed quite a bit which shows how unrealistic a clock of that speed would be. Slowing the 100MHz clock down a little bit more might be a bit more
foolproof, but the results show that is the correct clock speed to go with for these designs. The 500MHz might be able to be used for some of the smaller
designs if it is slowed down a bit, but if we had to choose one clock speed for all of the designs, it would be the 100MHz clock. All the numbers increase a lot
from the 100MHz design to the 500MHz design as expected. The faster clock speed means there will be more power consumption and a larger area due to an increased
number of buffers and flip flops to try and make timing. Even with all that there will still be timing violations, because you can only insert so many
buffers and flip flops to try and meet timing before it becomes unrealistic.

To run the scripts, the bash files provided in the respective directories will run each design. If you run it manually it will only run the design specified
in the set\_env.tcl file. The output directories are under the designs folder with a folder for each design.
\end{document}
